% Comments need first a "%"
% To write a %, a "\" in needed first
% To use a function is needed a \ first

\documentclass[a4paper, 12pt]{article} % function to determine the class of the document
% Use [ ] to personalize the font size and the paper size 
% There are so many ways to define the font size in small areas (see: https://it.overleaf.com/learn/latex/Font_sizes%2C_families%2C_and_styles )

\usepackage[utf8]{inputenc} % function to upload the packages
% Use [ ] to define the structures
\usepackage{graphicx} % package for graphics, u can use "\textcolor{}" function
\usepackage{color} % to use colors
\newcommand{\tr}{\textcolor{red}} % to create a new function to use faster a command
\usepackage{hyperref} % to link the references from text to figure or article
\usepackage{lineno}
\usepackage{listings} % use code in Latex
\usepackage{natbib} % to make a semi-automatic reference section
\usepackage{setspace} % to set the interline between objects
% Exists a package to modify automatically the references, but it's quiet complicated
\usepackage{natbib} % another package to facilitate the filling of bibliography 

\linenumbers % to insert the number of lines
% \doublespacing  to set a double interline (\usepackage{setspace} is needed)

\title{Esempio di articolo telerilevamento 2022} % function to create the title of the pdf
\author{Francesco Candini} % Author of the pdf
\date{July 2022} % Automatic (last compilation) or manual

\begin{document} % begin of the document, it needs an "end" in bottom

\maketitle

% Abstract and keywords
\begin{abstract}
   \centering This is a perfect abstract. 
\end{abstract}

\noindent \textbf{Keywords:  biodiversity; ecology; magic} % keywords alphabetically ordered
% \textbf{} to make font bold (grassetto)

\tableofcontents % after "\begin{document}" and "\maketitle"

% Abstract
% Keywords


\section{Introduction} \label{Sec.: Intro} % make a reference to the section Introduction


Bla bla bla. % with \\ it makes a white line before the rest of the text
% It's possible to use also the functions "\bigskip" or "\smallskip"

Lol, it's Latex time. % automatic indent, to remove it use "\noindent" function

\bigskip\noindent{\tiny{Look how small i am...}}

\textcolor{green}{Amazing}
\noindent{I} can joke showing you some \Large{\tr{really important data}}


\section{Study area}


\normalsize You Only Live Once, Y.O.L.O. 

Nice hvhuiauvivhuòhv (Figure \ref{Fig.: Magic}) and uvaguyahuegabr. % "\ref" to make references
% Function "\newpage" to use another page
\begin{figure}
 \includegraphics[width=0.35\textwidth]{magic the gathering.jpeg} 
 % problems with the size, so i modify the structures with []
 \centering % to center the image in the paper
 \caption{Nice pic!} % to write the caption (didascalia)
 \label{Fig.: Magic} % automatic label (etichetta)
\end{figure}


\section{Methods}


First I'll explain the formulas and then I will pass to the code.
Let's see a bulleted list: 
\begin{itemize}
    \item The \ref{Eq.: Newton}
    \item The \ref{Eq.: Fantasy}
    \item The code
\end{itemize}
Numerated list:
\begin{enumerate}
    \item The \ref{Eq.: Newton}
    \item The \ref{Eq.: Fantasy}
    \item The code (in subsection \ref{Code: 1})
\end{enumerate}


\subsection{Formulas}
Here is the formula used in this manuscript.
\begin{equation}
    F = G \frac{m_{1} \times m_2}{d^2} % subscript with _{} and apex with ^{} 
    % (not mandatory the {} if it's only 1 symbol)
    % function \times to write the x
    \label{Eq.: Newton}
\end{equation}

Let's invent an equation:
\begin{equation}
    A = \frac{\sqrt[3]{b \frac{c_{1} \times c_{2}}{\sqrt{d^{5\mu}}}}}{-\sum_{i}^{2000}{p(E) \times \log{p(E)}}} 
    % \sqrt[...] to choose the number, \sum to make a summation with apex and subscription
    % see on wiki how to write the greek letters
    \label{Eq.: Fantasy}
\end{equation}

\noindent Let's put a formula directly in the main text. 

\noindent We can apply this: $ F = G \frac{m_{1} \times m_2}{d^2} $


\subsection{Code}
Here is the code used in this manuscript.
\lstinputlisting[language = R]{R Narcissus tazetta.r}  


\section{Results}


These results were achieved according to Equation \ref{Eq.: Newton} and C


\section{Discussion}


As we've seen in section \ref{Sec.: Intro}

\noindent Our results are in line with \citep{Alternative_C._Gatti_2015}  % same use of \ref, but specific for citations

\noindent Bla bla bla, this thing is very important \citep{Alternative_C._Gatti_2015} %\citep{} 

\noindent \citet{Alternative_C._Gatti_2015} told us that this thing is very important %

\noindent \citet{Alternative_C._Gatti_2015, Sabatini_2018} % order of citations from older to the most recent!


\section{Conclusion}


\begin{thebibliography}{999} % 999 is a code 
\bibitem{C._Gatti_2015}
Cazzolla Gatti, R., Castaldi, S., Lindsell, J.A. et al. The impact of selective logging and clearcutting on forest structure, tree diversity and above-ground biomass of African tropical forests. Ecol Res 30, 119–132 (2015). \url{https://doi.org/10.1007/s11284-014-1217-3} % \url{} to make a link 
\bibitem[C. Gatti et al.(2015)]{Alternative_C._Gatti_2015}
Cazzolla Gatti, R., Castaldi, S., Lindsell, J.A. et al. The impact of selective logging and clearcutting on forest structure, tree diversity and above-ground biomass of African tropical forests. Ecol Res 30, 119–132 (2015). \url{https://doi.org/10.1007/s11284-014-1217-3}
\bibitem[Sabatini et al.(2018)]{Sabatini_2018}
Sabatini, FM, Burrascano, S, Keeton, WS, et al. Where are Europe’s last primary forests? Divers Distrib. 2018; 24: 1426– 1439. \url{https://doi.org/10.1111/ddi.12778}
\end{thebibliography}

\end{document}
